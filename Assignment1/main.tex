\documentclass{article}
\usepackage[utf8]{inputenc}
\usepackage[T1]{fontenc}
\usepackage{karnaugh-map}
\usetikzlibrary{calc}
\usepackage{placeins}

\usepackage{amsmath}
\usepackage{amsfonts}
\usepackage{multicol}
\usepackage{amssymb}
\usepackage[framed]{matlab-prettifier}
\usepackage{graphicx}
\usepackage[margin=0.75in]{geometry}
\usepackage{enumerate}
\usepackage{circuitikz}

\title{Assignment 1 | FPGA Lab}
\author{Tarun Sriteja ADipe(EE21MTECH11017)}
\date{January 2022} 

\begin{document}
\maketitle

\section{Question}

Derive a canonical POS expression for a boolean function F, represented by the following truth table :

 \begin{table}[h!]
  \begin{center}
    \begin{tabular}{|c|c|c|c|}
    \hline
      \textbf{P} & \textbf{Q} & \textbf{R} & \textbf{F(P,Q,R)}\\ 
      \hline
      0 & 0 & 0 & 0\\ \hline
      0 & 0 & 1 & 1\\  \hline
      0 & 1 & 0 & 1\\ \hline
      0 & 1 & 1 & 0\\ \hline
      1 & 0 & 0 & 0\\ \hline
      1 & 0 & 1 & 0\\ \hline
      1 & 1 & 0 & 1\\ \hline
      1 & 1 & 1 & 1\\ \hline
    \end{tabular}
  \end{center}
\end{table}


\section{Solution}

\subsection{POS from table}
from the truth table we can write  \\

    $F(P,Q,R) =  \prod(0,3,4,5)$\\
\subsection{Canonical POS Expression}
canonical POS can be written as \\

    $F(P,Q,R) =  (P+Q+R).(P+\overline{Q}+\overline{R}).(\overline{P}+Q+R). 
    (\overline{P}+Q+\overline{R})$


\subsection{Minimization using KMAPs}
Obtained POS expression can be minimized using a KMap. 

\begin{figure}[!ht]
\centering
\resizebox{\columnwidth}{!}
  {
    \begin{karnaugh-map}[4][2][1][Q+R][P]
    \minterms{1,2,6,7}
    \maxterms{0,3,4,5}
    \implicant{0}{4}
    \implicant{4}{5}
    \implicant{3}{3}
   
    \end{karnaugh-map}
}
\end{figure}
\FloatBarrier

\subsection{Minimized POS Expression}

$F = (Q+R).(\overline{P}+Q).(P+\overline{Q}+\overline{R})$


\subsection{NOR Logic Implementation}

Implementing it using NOR Logic :

\begin{center}
  $F = \overline{\overline{(Q+R).(\overline{P}+Q).(P+\overline{Q}+\overline{R})}}$\\
   \vspace{5pt}
   $F = \overline{\overline{(Q+R)}+\overline{(\overline{P}+Q)} +\overline{(P+\overline{Q}+\overline{R})}}$\\
   \vspace{5pt}
%      $F = \overline{{(\overline{P}.\overline{Q}.\overline{R})}+{(\overline{P}.Q.R)}+ {(P.\overline{Q}.\overline{R})}+ {(P.\overline{Q}.R)}}$\\
%   \vspace{5pt}
\end{center}
The expression can be implemented using all two input NOR Gates.

\begin{center}
\begin{circuitikz}
\draw
(0,2)node[nor port](mynor1){}
(0,4)node[nor port](mynor2){}
(0,6)node[nor port](mynor3){}
(2,0)node[nor port](mynor4){}
(2,2)node[nor port](mynor5){}
(2,5)node[nor port](mynor6){}
(4,1)node[nor port](mynor7){}
(4,5)node[nor port](mynor8){}
(6,1)node[nor port](mynor9){}
(6,5)node[nor port](mynor10){}
(8,3)node[nor port](mynor11){}

(mynor1.in 1)|-(mynor1.in 2)
(mynor2.in 1)|-(mynor2.in 2)
(mynor3.in 1)|-(mynor3.in 2)
(mynor1.out)|-(mynor5.in 1)
(mynor2.out)|-(mynor6.in 2)
(mynor3.out)|-(mynor6.in 1)
(mynor4.out)|-(mynor7.in 2)
(mynor5.out)|-(mynor7.in 1)
(mynor6.out)|-(mynor8.in 1)
(mynor8.out)|-(mynor10.in 2)
(mynor7.out)|-(mynor9.in 1)
(mynor8.in 1)|-(mynor8.in 2)
(mynor9.in 1)|-(mynor9.in 2)
(mynor9.out)|-(mynor11.in 2)
(mynor10.out)|-(mynor11.in 1)


(mynor1.in 1) node (A1)     [anchor=east]           {P}
(mynor2.in 1) node (A2)     [anchor=east]           {Q}
(mynor3.in 1) node (A3)     [anchor=east]           {R}
(mynor4.in 1) node (A4)     [anchor=east]           {R}
(mynor4.in 2) node (A5)     [anchor=east]           {Q}
(mynor5.in 2) node (A6)     [anchor=east]           {Q}
(mynor10.in 1) node (A7)     [anchor=east]           {P}
(mynor11.out) node (A8)     [anchor=east, xshift=0.5cm]           {F};

\end{circuitikz}
\end{center}

\end{document}